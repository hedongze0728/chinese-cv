\resheading{教育经历}
  \begin{itemize}[leftmargin=*]
     \item
      \ressubheading{中国科学院高能物理研究所}{联合培养}{粒子天体中心}{2016.10 -- 至今}  
      {\small
      \begin{itemize}
         \item 通过模拟观测数据预期LHAASO实验对重暗物质粒子湮灭与衰变性质的探测前景     
      \end{itemize}
      }
    \item
      \ressubheading{东北大学}{博士在读}{理学院$\cdot$理论物理}{2016.09 -- 至今}  
      {\small
      \begin{itemize}
         \item 研究方向:  暗物质间接探测、宇宙线物理
         \item 毕业论文:《暗物质间接探测与宇宙线传播若干问题的研究》(拟定)
      \end{itemize}
      }
      
     \item    
      \ressubheading{东北大学}{硕士学位}{理学院$\cdot$光学}{2014.09 -- 2016.06}
      {\small
      \begin{itemize}
         \item{研究方向:  宇宙学}
         \item 专业课程: 广义相对论、现代宇宙学、高等量子力学、量子统计、固体理论等       
         \item 毕业论文:《红移漂移模拟数据对宇宙学模型参数的限制》
      \end{itemize}
      }

    \item
      \ressubheading{沈阳师范大学}{学士学位}{物理科学与技术学院$\cdot$物理学(师范)}{2010.09 -- 2014.06}
      {\small
      \begin{itemize}
         \item 专业课程: 力学、热学、电磁学、光学、原子物理学、理论力学、电动力学、热力学与统计物理学、量子力学、固体物理等
      \end{itemize}
      }
  \end{itemize}